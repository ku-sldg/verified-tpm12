\documentclass[10pt]{article}

%\usepackage{hyperref}
\usepackage{semantic}
\usepackage{listings}
\usepackage{times}
\usepackage[scaled]{beramono}
\usepackage[T1]{fontenc}
\usepackage{fullpage}
\usepackage{alltt}
\usepackage{natbib}
\usepackage{graphicx}
\usepackage{url}
\usepackage{fancyhdr}
\usepackage{tikz}
\usetikzlibrary{arrows,shadows}
\usepackage[underline=false]{pgf-umlsd}
\usepackage{trust}
\pagestyle{fancy}

\lstdefinelanguage{pvs}{
   basicstyle=\color{black}\ttfamily,
   emphstyle=\bfseries,
   keywordstyle=\bfseries,
   commentstyle=\color{red},
   stringstyle=\color{blue},
% identifierstyle=\sffamily,
   morekeywords={theory,begin,end,type,type+,axiom,theorem,lemma,datatype,
     judgement,assumption,assuming,endassuming,lambda,forall,exists,if,then,
     else,endif},
   emph={true,false,not,and,or,xor,iff,implies,lambda,forall,exists},
   mathescape=true,
   sensitive=false,
   showstringspaces=false,
   morestring=[b]",
 }

\lstdefinelanguage{logic}{
  basicstyle=\color{black}\sffamily,
  emphstyle=\bfseries,
  keywordstyle=\bfseries,
  commentstyle=\color{red},
  stringstyle=\color{blue},
  morekeywords={theory,begin,end,if,then,else,endif},
  emph={true,false,not,and,or,xor,iff,implies,lambda,forall,exists},
  sensitive=false,
  showstringspaces=false,
  mathescape=true,
  literate={->}{{$\rightarrow$}}1 {=>}{{$\Rightarrow$}}1 
            {forall}{{$\forall$}}1 {EXISTS}{{$\exists$}}1
            {lambda}{{$\lambda$}}1 {<=>}{{$\Rightleftarrow$}}1
            {==}{{$\equiv$}}1,
  morestring=[b]",
 }

\lstset{language=pvs}

\lhead{}
\rhead{}
\lfoot{\copyright The University of Kansas, 2012}
\cfoot{\thepage}


\newtheorem{conjecture}{Conjecture}
\newtheorem{obligation}{Obligation}
\newtheorem{definition}{Definition}


\usepackage{ifthen}
\newboolean{submission}  %%set to true for the submission version
\setboolean{submission}{false}
%\setboolean{submission}{true}
\ifthenelse
{\boolean{submission}}
{ \newcommand{\todo}[1]{ } } % hide todo
{ \newcommand{\todo}[1]{ % show todo
   \marginpar{\raggedright\footnotesize{#1}}
               }}

\newcommand{\squeeze}{\parskip=0pt\itemsep=0pt}

%\parskip=\medskipamount
%\parindent=0pt

\bibliographystyle{abbrvnat}

\title{vTPM Modeling Notes}
\author{Perry Alexander and Brigid Halling \\
  Information and Telecommunication Technology Center \\
  The University of Kansas \\
  \url{{palexand,bhalling}@ku.edu} \\
  \\
  Allen Goldberg, Eric Smith, Alessandro Coglio \\
  The Kestrel Institute \\
  \url{{addresses}@kestrel.edu}}

\begin{document}

\maketitle
\tableofcontents
\listoffigures
\listoftables

\newtheorem{invariant}{Invariant}
\newtheorem{post}{Postcondition}
\newtheorem{pre}{Precondition}

\begin{abstract}
  This document is designed to document our work understanding the
  interaction among elements of the vTPM infrastructure and
  interaction between the vTPM and its operational environment.
\end{abstract}

\section{Introduction}

\section{vTPM Manager Data}

Data objects used by the vTPM subsystem are listed in
figure~\ref{fig:vtpm-information-table-format} and
figure~\ref{fig:other-data}.
Figure~\ref{fig:vtpm-information-table-format} lists data that is
stored by the vTPM Manager and includes PCR contents associated with
vTPM and controller measurements, long- and short-term IDs, the sealed
vTPM Data key and vTPM Data hash.

\begin{figure}[hbtp]
  \centering
  \begin{tabular}{llll}
    \emph{Field} & \emph{Description} & \emph{Initialized} &
    \emph{Updated} \\ \hline
    $ID_{vtpm}$ & The persistent name of the vTPM & provisioning &
    never \\
    $ID_{dom}$ & Domain ID of vTPM VM & build & build \\
    $ID_{con}$ & The persistent name of the vTPM's controller &
    provisioning & never \\
    $\hash{\encrypt{D_{vtpm}}{K_3}}$ & vTPM data hash value & provisioning & vTPM sleep \\
    $PCR_6$ & Hash of the vTPM + Long Term Name & provisioning & boot
    \\
    $PCR_5$ & Hash of the platform controller & provisioning & boot \\
    $\seal{K_{3}}{PCR_5,PCR_6}$ & Sealed symmetric key encrypting vTPM data &
    provisioning & vTPM sleep \\
  \end{tabular}
  \caption{vTPM Information Table contents}
  \label{fig:vtpm-information-table-format}
\end{figure}

Figure~\ref{fig:other-data} lists data that is involved in vTPM
related communications and includes the encrypted vTPM data, the vTPM
image, and hashes of the controller and booted vTPM.  Note that the
hash values are called out separately here as they may differ from
what is stored in the vTPM Manager.

\begin{figure}[hbtp]
  \centering
  \begin{tabular}{lllll}
    \emph{Data} & \emph{Description} & \emph{Location} &
    \emph{Initialized} & \emph{Modified} \\ \hline
    $\encrypt{D_{vtpm}}{K_{3}}$ & Encrypted vTPM data & Host Storage &
    provisioning & vTPM sleep \\
    $I_{vtpm}$ & \emph{vTPM Image} & Host Storage & provisioning & never
    \\
    $\hash{(vTPM+ID_{vtpm})}$ & Hash of vTPM named $ID_{vtpm}$ & vTPM
    Manager & build & build \\ 
    $\hash{Controller}$ & Hash of Controller and Schema & vTPM
    Manager & build & build \\ 
  \end{tabular}
  \caption{Other data used during vTPM activities}
  \label{fig:other-data}
\end{figure}

\section{Provisioning Sequence}

I'm not at all sure I have the provisioning sequence right.  The vTPM
can be used to generate an $EK$ while the \textsf{TPM\_TakeOwnership}
command can be used to generate an $SRK$ after startup.  What else is
initialized?  Monotonic counter values and other stuff not specified here.

\begin{enumerate}
\item Platform booted through Controller -- $PCR_5$ and $ID_{con}$ now known by
  the vTPM Manager
\item A new Long Term Name is generated for the new vTPM
\item The new vTPM is built using the Domain Builder -- $PCR_6$ is now
  known by the vTPM Manager
\item vTPM data is initialized -- including fresh EK value.
\item vTPM is put to sleep using the standard sleep protocol
\end{enumerate}

\subsection*{Pre-conditions for Provisioning}

\begin{pre}[MLE Correctness]
  Running MLE components are the anticipated components.
\end{pre}

\begin{pre}[MLE Measurements]
  hTPM logical $PCR_{0-4}$ contain hashes of running MLE components.
\end{pre}

\begin{pre}[vTPM Data Hash]
  hTPM logical $PCR_5$ contains a hash of the controller/schema pair
  uniquely identifying the vTPM's virtual platform.\footnote{The
    controller and schema may be hashed separately with the schema
    providing the unique identifier for the platform.}
\end{pre}

\subsection*{Post-conditions for Provisioning}

\subsubsection*{vTPM Information Table}

Conditions on the vTPM information table residing in the vTPM
Manager:\footnote{Note that we use the PVS notation $X(ID_{vtpm}$ to
  reference property $X$ from the vTPM Manager's record for vTPM $ID_{vtpm}$.}


\begin{post}[Controller Measurement]
  The vTPM's controller measurement is available in the vTPM
  information table referenced as $PCR_5$.

  \[PCR_5(ID_{vtpm}) = \hash{Controller}\]

\end{post}

\begin{post}[vTPM Measurement]
  The vTPM's measurement is unique and available in the vTPM
  information table, referenced as $PCR_6$.

  \[PCR_6(ID_{vtpm}) = \hash{(vTPM+ID_{vtpm})}\]

\end{post}

\begin{post}[vTPM Data Existence]
  The vTPM's initial data exists and is available through its table entry.

  \[\forall i:ID_{vtpm} \cdot \exists d:D_{vtpm} \cdot \encrypt{d}{K_{d}(ID_{vtpm})}\]
\end{post}

\begin{post}[vTPM Data Seal]
  The vTPM data session key will be sealed to hTPM logical $PCR_{0-6}$
  indirectly sealing the vTPM data to logical $PCR_{0-6}$.
\end{post}

\begin{post}[vTPM Data Hash]
  The vTPM's data hash will be stored in the vTPM information table.
\end{post}

\begin{post}[Endorsement Key]
  The vTPM's Endorsement Key (EK) is initialized and a certificate
  binding it to its vTPM is available.

  \[\exists k:KEY \cdot EK(ID_{vtpm})=k\] 
  \[\exists \cert{EK(D_{vtpm})}{CA}:cert \cdot TRUE\]
\end{post}

\begin{post}[Unique vTPM Name]
  The $ID_{vtpm}$ will be unique.

  \[\forall i,j:ID_{vtpm} \cdot t(i)=t(j) <=> i=j\]
\end{post}

\subsection{Invariants Following Provisioning}

\subsection{Conjectures}

\begin{conjecture}[vTPM Data Seal]
  A vTPM cannot access its data if its platform was not built from the
  correct parts.

  The state of a vTPM's data session key following provisioning or
  sleep is sealed to $PCR_0$ through $PCR_6$ that contain measurements
  of MLE and SVP components that boot before the vTPM:

  \[\seal{K_{3}}{PCR_0\ldots PCR_6}\]

  Given the key is initially sealed during provisioning to $PCR$s
  resulting from hashing correct components, $K_{3}$ will not unseal
  unless: (i) the unsealing TPM is the same as the sealing TPM; and
  (ii) hashes in sealing $PCR$s must match hashes of those in the TPM
  at provisioning.  Assumption (i) and a certificate binding the
  sealing TPM to a system ensures that the $K_{3}$ will only unseal on
  the hardware it was sealed on.  Assumption (ii) and the assumptions
  that the hash function is injective and the MLE is correct ensures
  that $K_{3}$ will only unseal if the correct Domain Builder, vTPM
  Manager, Controller and Schema, and vTPM were started during system
  boot.

  The assumption that $ID_{vtpm}$ values are unique ensures that $K_{3}$
  will not unseal unless the correct vTPM makes the unseal request.
  The vTPM Manager resets $PCR_5$ and $PCR_6$ based on the domain ID
  ($ID_{dom}$) of vTPM initiating communication.
\end{conjecture}

\section{Startup Sequence}

Establishing an operational vTPM involves getting the vTPM up and
running, then installing its data.  All vTPM's must have an associated
Controller running before they can start.  Like vTPM information,
Controller information is maintained by the vTPM Manager.  The process
of registering a Controller with the vTPM Manager and starting a new
vTPM are outlined in the following to subsections.

\subsection{Controller Running}

Figure~\ref{fig:data-controller} describes the sequence of events
necessary for a Controller to initialize it's data in the vTPM
Manager's vTPM information table.  When a Controller is started by
either the SVP Controller (documented) or by the Domain Builder as a
part of the SVP boot process, information in the vTPM Manager must be
updated to record the hash associated with the Controller's ID.
Following is a textual description of the process:

\begin{enumerate}
  \squeeze
%  \parskip=0pt\itemsep=0pt
\item Controller (UVP) or Domain Builder (SVP) requests a build by
  sending a manifest identifying the Controller image and its Long
  Term Name to the Domain Builder.
\item Domain Builder hashes the Controller image and Long Term Name
\item Domain Builder starts the Controller VM resulting in a new
  Domain ID
\item Domain Builder sends the Long Term Name, hash, and Controller
  Name to the vTPM Manager
\item vTPM Manager updates data associated with all entries
  referencing the Controller Name in the vTPM information table
\item Domain Builder returns the new Controller measurement to the
  initiating controller
\end{enumerate}

Note that this sequence differs modestly when the SVP controller is
started.  The Domain Builder initiates the process itself and there is
no Controller to receive the resulting hash.

\begin{figure}[hbtp]
  \centering
  \begin{sequencediagram}
  \newthread[white]{controller}{Controller}
  \newinst[2.0]{db}{Domain Builder} 
  \newinst[2.0]{manager}{vTPM Manager}
 \begin{call}{controller}{$I_{con},ID_{con}$}{db}{$\hash{(I_{con}+ID_{con})}$}
   \begin{callself}{db}{build $I_{con},ID_{con}$}{$\hash{(I_{con}+ID_{con})}$}
   \end{callself}
   \begin{call}{db}{$ID_{con},\hash{(I_{con}+ID_{con})}$}{manager}{}
     \begin{callself}{manager}{Update $ID_{con}$ entry}{}
     \end{callself}
   \end{call}
 \end{call}
  \end{sequencediagram}
  \caption{Controller data initialization.}
  \label{fig:data-controller}
\end{figure}

\subsection{vTPM Running}

Figure~\ref{fig:running-vtpm} describes the sequence of events
necessary for a vTPM to start.  The result of this sequence is a vTPM
running and the vTPM Manager table reflecting vTPM information.  The
vTPM will not have its data yet.  Following is a textual description
of the process:

\begin{enumerate}
  \parskip=0pt\itemsep=0pt
\item Controller requests a build by sending a manifest identifying
  the vTPM image and its Long Term Name to the Domain Builder
\item Domain Builder hashes the vTPM image and Long Term Name
\item Domain Builder starts the vTPM VM resulting in a new Domain ID
\item Domain Builder sends the Long Term Name, Hash, Domain ID, and
  Controller Name to the vTPM Manager
\item vTPM Manager updates data associated with the Long Term Name and
  Controller Name in the vTPM information table
\end{enumerate}

\begin{figure}
\begin{sequencediagram}
  \newthread[white]{controller}{Controller}
  \newinst[1.5]{db}{Domain Builder}
  \newinst[4.5]{manager}{vTPM Manager}
  
  \begin{call}{controller}{$I_{vtpm}$,$ID_{vtpm}$}{db}{$\hash{(I_{vtpm}+ID_{vtpm})}$}
    \begin{callself}{db}{build $I_{vtpm},ID_{vtpm}$}{$\hash{(I_{vtpm}+ID_{vtpm}),ID_{dom}}$}\end{callself}
    \begin{call}{db}{$\hash{(I_{vtpm}+ID_{vtpm})}$,$ID_{vtpm}$,$ID_{dom}$}{manager}{}
      \begin{callself}{manager}{Update $ID_{vtpm}$ entry}{}\end{callself}
    \end{call}
  \end{call}
\end{sequencediagram}
\caption{Command sequence to establish a running vTPM without its
  data.}
\label{fig:running-vtpm}
\end{figure}

\subsection{vTPM Data Initialization}

Figure~\ref{fig:data-vtpm} describes the sequence of events necessary
for a vTPM to load its data.  The result of this sequence is a vTPM
ready to provide TPM services to its associated virtual platform.
Following is a textual description of the process:

\begin{enumerate}
  \parskip=0pt\itemsep=0pt
\item vTPM requests its data key from the vTPM Manager
\item vTPM Manager uses the vTPM Domain ID to find table entry
\item vTPM Manager resets $PCR_5$ and $PCR_6$ with table entry
  values\footnote{This happens whenever the vTPM Manager communicates
    with a vTPM.}
\item vTPM Manager attempts to unseal $K_{3}$
  \begin{enumerate}
    \parskip=0pt\itemsep=0pt
  \item On failure abort the request response
  \item On success return $K_{3}$ and $\hash{\encrypt{D_{vtpm}}{K_3}}$ to vTPM
  \end{enumerate}
\item vTPM requests its encrypted data from Host Storage
\item vTPM decrypts its data with $K_{3}$ and checks hash against
  $\hash{\encrypt{D_{vtpm}}{K_3}}$
\item vTPM installs data and begins providing services.
\end{enumerate}

\begin{figure}
\begin{sequencediagram}
  \newthread[white]{vtpm}{vTPM}
  \newinst[2.0]{manager}{vTPM Manager}
  \newinst[2.5]{tpm}{hTPM}
  \newinst[2.0]{store}{Host Storage}
  \begin{call}{vtpm}{$ID_{dom}$}{manager}{$K_{3}$}
    \begin{call}{manager}{Reset $PCR_5$}{tpm}{}
    \end{call}
    \begin{call}{manager}{Reset $PCR_6$}{tpm}{}
    \end{call}
    \begin{call}{manager}{$\seal{K_{3}}{PCR_5,PCR_6}$}{tpm}{$K_{3}$}
      \begin{callself}{tpm}{$unseal$}{$K_{3}$}\end{callself}
    \end{call}
  \end{call}
  \begin{call}{vtpm}{Request Data}{store}{$\encrypt{D_{vtpm}}{K_{3}}$}
  \end{call}
  \begin{callself}{vtpm}{$\encrypt{D_{vtpm}}{K_{3}}$}{$D_{vtpm}$}\end{callself}
\end{sequencediagram}
\caption{Command sequence to load vTPM data.}
\label{fig:data-vtpm}
\end{figure}

\section{Sleep Sequence}

Figure~\ref{fig:sleep-vtpm} describes the sequence of events necessary
for putting a vTPM to sleep.  The result of this sequence is the
vTPM's data saved to Host Storage encrypted with a fresh session key.
The vTPM Manager's table will be updated to reflect the new key and
data hash.  Following is a textual description of the process:

\begin{enumerate}
  \parskip=0pt\itemsep=0pt
\item vTPM generates a fresh session key, $K_{3}$
\item vTPM hashes its data, $\hash{\encrypt{D_{vtpm}}{K_3}}$ and encrypts with $K_{3}$
\item vTPM stores $\encrypt{D_{vtpm}}{K_{3}}$ in Host Storage
\item vTPM sends sleep request to vTPM manager with $K_{3}$ and
  $\hash{\encrypt{D_{vtpm}}{K_3}}$
\item vTPM Manager uses vTPM Domain ID to find table entry
\item vTPM Manager resets and loads $PCR_5$ and $PCR_6$ from table
  entry
\item vTPM Manager seals $K_{3}$ to TPM state
\item vTPM Manager updates its table with
  $\seal{K_{3}}{PCR_5,PCR_6}$ and $\hash{\encrypt{D_{vtpm}}{K_3}}$
\item vTPM Manager clears Domain ID\footnote{I'm not certain of this,
    but it seems reasonable}
\end{enumerate}

\begin{figure}
\begin{sequencediagram}
  \newthread[white]{vtpm}{vTPM}
  \newinst[2.0]{manager}{vTPM Manager}
  \newinst[2.5]{tpm}{hTPM}
  \newinst[2.0]{store}{Host Storage}
  \begin{callself}{vtpm}{Generate $K_{3}'$}{$K_{3}'$}\end{callself}
  \begin{callself}{vtpm}{Hash $D_{vtpm}$}{$\hash{\encrypt{D_{vtpm}}{K_3}}$}\end{callself}
  \begin{callself}{vtpm}{Encrypt $D_{vtpm}$}{$\encrypt{D_{vtpm}}{K_{3}'}$}\end{callself}  
  \begin{call}{vtpm}{$ID_{dom},K_{3},\hash{\encrypt{D_{vtpm}}{K_3}}$}{manager}{}
    \begin{call}{manager}{Reset $PCR_5$}{tpm}{}
    \end{call}
    \begin{call}{manager}{Reset $PCR_6$}{tpm}{}
    \end{call}
    \begin{call}{manager}{$K_{3}'$}{tpm}{$\seal{K_{3}'}{PCR_5,PCR_6}$}
      \begin{callself}{tpm}{$seal$}{$\seal{K_{3}'}{PCR_5,PCR_6}$}\end{callself}
    \end{call}
    \begin{callself}{manager}{Update $ID_{dom}$ entry}{}\end{callself}
  \end{call}
  \begin{call}{vtpm}{\encrypt{D_{vtpm}}{K_{3}'}}{store}{}
  \end{call}
\end{sequencediagram}
\caption{Command sequence to put a vTPM to sleep.}
\label{fig:sleep-vtpm}
\end{figure}
\bibliography{vtpm}

\end{document}
