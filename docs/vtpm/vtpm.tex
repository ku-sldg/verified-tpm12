\documentclass[10pt]{article}

%\usepackage{hyperref}
\usepackage{fullpage}
\usepackage{alltt}
\usepackage{natbib}
\usepackage{graphicx}
\usepackage{url}
\usepackage{fancyhdr}
\usepackage{tikz}
\usetikzlibrary{arrows,shadows}
\usepackage[underline=false]{pgf-umlsd}
\usepackage{trust}
\pagestyle{fancy}

\lhead{}
\rhead{}
\lfoot{\copyright The University of Kansas, 2012}
\cfoot{\thepage}


\newtheorem{conjecture}{Conjecture}
\newtheorem{obligation}{Obligation}
\newtheorem{definition}{Definition}


\usepackage{ifthen}
\newboolean{submission}  %%set to true for the submission version
\setboolean{submission}{false}
%\setboolean{submission}{true}
\ifthenelse
{\boolean{submission}}
{ \newcommand{\todo}[1]{ } } % hide todo
{ \newcommand{\todo}[1]{ % show todo
   \marginpar{\raggedright\footnotesize{#1}}
               }}

\parskip=\medskipamount
\parindent=0pt

\bibliographystyle{abbrvnat}

\title{vTPM Modeling Notes}
\author{Perry Alexander and Brigid Halling \\
  Information and Telecommunication Technology Center \\
  The University of Kansas \\
  \url{{palexand,bhalling}@ku.edu} \\
  \\
  Allen Goldberg, Eric Smith, Alessandro Coglio \\
  The Kestrel Institute \\
  \url{{addresses}@kestrel.edu}}

\begin{document}

\maketitle
\tableofcontents
\listoffigures
\listoftables

\begin{abstract}
  This document is designed to document our work understanding the
  interaction among elements of the vTPM infrastructure and
  interaction between the vTPM and its operational environment.
\end{abstract}

\section{Introduction}

\section{vTPM Manager Data}

Data objects used by the vTPM subsystem are listed in
figure~\ref{fig:vtpm-information-table-format} and
figure~\ref{fig:other-data}.
Figure~\ref{fig:vtpm-information-table-format} lists data that is
stored by the vTPM Manager and includes PCR contents associated with
vTPM and controller measurements, long- and short-term IDs, the sealed
vTPM Data key and vTPM Data hash.

\begin{figure}[hbtp]
  \centering
  \begin{tabular}{llll}
    \emph{Field} & \emph{Description} & \emph{Initialized} &
    \emph{Updated} \\ \hline
    $ID_{vtpm}$ & The persistent name of the vTPM & provisioning &
    never \\
    $ID_{dom}$ & Domain ID of vTPM VM & build & build \\
    $\hash{D}$ & vTPM data hash value & provisioning & vTPM sleep \\
    $PCR_6$ & Hash of the vTPM + Long Term Name & provisioning & boot
    \\
    $PCR_5$ & Hash of the platform controller & provisioning & boot \\
    $\seal{K_D}{PCR_5,PCR_6}$ & Sealed symmetric key encrypting vTPM data &
    provisioning & vTPM sleep \\
  \end{tabular}
  \caption{vTPM Information Table contents}
  \label{fig:vtpm-information-table-format}
\end{figure}

Figure~\ref{fig:other-data} lists data that is involved in vTPM
related communications and includes the encrypted vTPM data, the vTPM
image, and hashes of the controller and booted vTPM.  Note that the
hash values are called out separately here as they may differ from
what is stored in the vTPM Manager.

\begin{figure}[hbtp]
  \centering
  \begin{tabular}{lllll}
    \emph{Data} & \emph{Description} & \emph{Location} &
    \emph{Initialized} & \emph{Modified} \\ \hline
    $\encrypt{D}{K_D}$ & Encrypted vTPM data & Host Storage &
    provisioning & vTPM sleep \\
    $I_{vtpm}$ & \emph{vTPM Image} & Host Storage & provisioning & never
    \\
    $\hash{(vTPM+ID_{vtpm})}$ & Hash of vTPM named $ID_{vtpm}$ & vTPM
    Manager & build & build \\ 
    $\hash{Controller}$ & Hash of Controller and Schema & vTPM
    Manager & build & build \\ 
  \end{tabular}
  \caption{Other data used during vTPM activities}
  \label{fig:other-data}
\end{figure}

\section{Provisioning Sequence}

I'm not at all sure I have the provisioning sequence right.  The vTPM
can be used to generate an $EK$ while the \textsf{TPM\_TakeOwnership}
command can be used to generate an $SRK$ after startup.  What else is
initialized?  Monotonic counter values and other stuff not specified here.

\begin{enumerate}
\item Platform booted through Controller -- $PCR_5$ is now known by
  the vTPM Manager
\item A new Long Term Name is generated for the new vTPM
\item The new vTPM is built using the Domain Builder -- $PCR_6$ is now
  known by the vTPM Manager
\item vTPM data is initialized -- including fresh EK value.
\item vTPM is put to sleep using the standard sleep protocol
\end{enumerate}

\section{Startup Sequence}

Establishing an operational vTPM involves getting the vTPM up and
running, then installing its data.  These are outlined in the
following to subsections.

\subsection{vTPM Running}

Figure~\ref{fig:running-vtpm} describes the sequence of events
necessary for a vTPM to start.  The result of this sequence is a vTPM
running and the vTPM Manager table reflecting vTPM information.  The
vTPM will not have its data yet.  Following is a textual description
of the process:

\begin{enumerate}
  \parskip=0pt\itemsep=0pt
\item Controller requests a build by sending a manifest identifying
  the vTPM image and its Long Term Name to the Domain Builder
\item Domain Builder hashes the vTPM image and Long Term Name
\item Domain Builder starts the vTPM VM resulting in a Domain ID
\item Domain Builder sends the Long Term Name, Hash, and Domain ID to
  the vTPM Manager
\item vTPM Manager updates data associated with the Long Term Name in
  the vTPM information table
\end{enumerate}

\begin{figure}
\begin{sequencediagram}
  \newthread[white]{controller}{Controller}
  \newinst[1.5]{db}{Domain Builder}
  \newinst[4.5]{manager}{vTPM Manager}
  
  \begin{call}{controller}{$I_{vtpm}$,$ID_{vtpm}$}{db}{$\hash{(I_{vtpm}+ID_{vtpm})}$}
    \begin{callself}{db}{build vTPM,$ID_{vtpm}$}{$\hash{(I_{vtpm}+ID_{vtpm}),ID_{dom}}$}\end{callself}
    \begin{call}{db}{$\hash{(I_{vtpm}+ID_{vtpm})}$,$ID_{vtpm}$,$ID_{dom}$}{manager}{}
      \begin{callself}{manager}{Update $ID_{vtpm}$ entry}{}\end{callself}
    \end{call}
  \end{call}
\end{sequencediagram}
\caption{Command sequence to establish a running vTPM without its
  data.}
\label{fig:running-vtpm}
\end{figure}

\subsection{vTPM Data Initialization}

Figure~\ref{fig:data-vtpm} describes the sequence of events
necessary for a vTPM to load its data.  The result of this sequence is a vTPM
ready to provide TPM services to its associated virtual platform.  Following is a textual description
of the process:

\begin{enumerate}
  \parskip=0pt\itemsep=0pt
\item vTPM requests its data key from the vTPM Manager
\item vTPM Manager uses the vTPM Domain ID to find table entry
\item vTPM Manager resets $PCR_5$ and $PCR_6$ with table entry
  values\footnote{This happens whenever the vTPM Manager communicates
    with a vTPM.}
\item vTPM Manager attempts to unseal $K_D$
  \begin{enumerate}
    \parskip=0pt\itemsep=0pt
  \item On failure abort the request response
  \item On success return $K_D$ and $\hash{D}$ to vTPM
  \end{enumerate}
\item vTPM requests its encrypted data from Host Storage
\item vTPM decrypts its data with $K_D$ and checks hash against
  $\hash{D}$
\item vTPM installs data and begins providing services.
\end{enumerate}

\begin{figure}
\begin{sequencediagram}
  \newthread[white]{vtpm}{vTPM}
  \newinst[2.0]{manager}{vTPM Manager}
  \newinst[2.5]{tpm}{hTPM}
  \newinst[2.0]{store}{Host Storage}
  \begin{call}{vtpm}{$ID_{dom}$}{manager}{$K_D$}
    \begin{call}{manager}{Reset $PCR_5$}{tpm}{}
    \end{call}
    \begin{call}{manager}{Reset $PCR_6$}{tpm}{}
    \end{call}
    \begin{call}{manager}{$\seal{K_D}{PCR_5,PCR_6}$}{tpm}{$K_D$}
      \begin{callself}{tpm}{$unseal$}{$K_D$}\end{callself}
    \end{call}
  \end{call}
  \begin{call}{vtpm}{Request Data}{store}{$\encrypt{D}{K_D}$}
  \end{call}
  \begin{callself}{vtpm}{$\encrypt{D}{K_D}$}{D}\end{callself}
\end{sequencediagram}
\caption{Command sequence to load vTPM data.}
\label{fig:data-vtpm}
\end{figure}

\section{Sleep Sequence}

Figure~\ref{fig:sleep-vtpm} describes the sequence of events necessary
for putting a vTPM to sleep.  The result of this sequence is the
vTPM's data saved to Host Storage encrypted with a fresh session key.
The vTPM Manager's table will be updated to reflect the new key and
data hash.  Following is a textual description of the process:

\begin{enumerate}
  \parskip=0pt\itemsep=0pt
\item vTPM generates a fresh session key, $K_D$
\item vTPM hashes its data, $\hash{D}$ and encrypts with $K_D$
\item vTPM stores $\encrypt{D}{K_D}$ in Host Storage
\item vTPM sends sleep request to vTPM manager with $K_D$ and
  $\hash{D}$
\item vTPM Manager uses vTPM Domain ID to find table entry
\item vTPM Manager resets and loads $PCR_5$ and $PCR_6$ from table
  entry
\item vTPM Manager seals $K_D$ to TPM state
\item vTPM Manager updates its table with
  $\seal{K_D}{PCR_5,PCR_6}$ and $\hash{K_D}$
\item vTPM Manager clears Domain ID\footnote{I'm not certain of this,
    but it seems reasonable}
\end{enumerate}

\begin{figure}
\begin{sequencediagram}
  \newthread[white]{vtpm}{vTPM}
  \newinst[2.0]{manager}{vTPM Manager}
  \newinst[2.5]{tpm}{hTPM}
  \newinst[2.0]{store}{Host Storage}
  \begin{callself}{vtpm}{Generate $K_D'$}{$K_D'$}\end{callself}
  \begin{callself}{vtpm}{Hash $D$}{$\hash{D}$}\end{callself}
  \begin{callself}{vtpm}{Encrypt $D$}{$\encrypt{D}{K_D'}$}\end{callself}  
  \begin{call}{vtpm}{$ID_{dom},K_D,\hash{D}$}{manager}{}
    \begin{call}{manager}{Reset $PCR_5$}{tpm}{}
    \end{call}
    \begin{call}{manager}{Reset $PCR_6$}{tpm}{}
    \end{call}
    \begin{call}{manager}{$K_D'$}{tpm}{$\seal{K_D'}{PCR_5,PCR_6}$}
      \begin{callself}{tpm}{$seal$}{$\seal{K_D'}{PCR_5,PCR_6}$}\end{callself}
    \end{call}
    \begin{callself}{manager}{Update $ID_{dom}$ entry}{}\end{callself}
  \end{call}
  \begin{call}{vtpm}{\encrypt{D}{K_{D}'}}{store}{}
  \end{call}
\end{sequencediagram}
\caption{Command sequence to put a vTPM to sleep.}
\label{fig:sleep-vtpm}
\end{figure}
\bibliography{vtpm}

\end{document}
