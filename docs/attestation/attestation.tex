\documentclass[10pt]{article}

%\usepackage{hyperref}
\usepackage{fullpage}
\usepackage{alltt}
\usepackage{natbib}
\usepackage{graphicx}
\usepackage{url}
\usepackage{fancyhdr}
\usepackage{tikz}
\usetikzlibrary{arrows,shadows}
\usepackage[underline=false]{pgf-umlsd}
\usepackage{trust}
\pagestyle{fancy}

\lhead{}
\rhead{}
\lfoot{\copyright The University of Kansas, 2012}
\cfoot{\thepage}


\newtheorem{conjecture}{Conjecture}
\newtheorem{obligation}{Obligation}
\newtheorem{definition}{Definition}


\usepackage{ifthen}
\newboolean{submission}  %%set to true for the submission version
\setboolean{submission}{false}
%\setboolean{submission}{true}
\ifthenelse
{\boolean{submission}}
{ \newcommand{\todo}[1]{ } } % hide todo
{ \newcommand{\todo}[1]{ % show todo
   \marginpar{\raggedright\footnotesize{#1}}
               }}

\parskip=\medskipamount
\parindent=0pt

\bibliographystyle{abbrvnat}

\title{Attestation Protocols: A Tutortial Introduction}
\author{Perry Alexander \and Brigid Halling}

\begin{document}

\maketitle
\tableofcontents
\listoffigures
\listoftables

\begin{abstract}
  This document is intended to provide a tutorial overview of the
  basic attestation protocols used by a TPM.
\end{abstract}

\section{Introduction}

\section{Privacy Certificate Authority Based Attestation}

There are two versions of the Privacy CA attestation process, one
documented by~\citet{Ryan:09:Introduction-to} and the other documented
in the TCG specification \citep{---::TCG-TPM-Specifi}.  They are not
dramatically different, but should be reconciled.

\subsection{The Ryan Approach}

The Ryan protocol, shown in figure~\ref{fig:ryan-ca}, is documented in
his invaluable technical report~\cite{Ryan:09:Introduction-to}.  We
have enhanced it here to include more explicit interaction between the
appraiser and the user software.  

\begin{enumerate}
  \parskip=0pt\itemsep=0pt
\item An appraiser sends an attestation request indicating what PCRs
  are needed using a $PCR$ make along with a nonce, $n$.  (This request
  is not formally documented, but its specifics are not critical for
  this discussion.)

\item The user software receives the request and requests a fresh AIK
  from the TPM, wrapped with the TPM's SRK using the
  \verb+TPM_MakeIdentity+ command.  Parameters to the
  \verb+TPM_MakeIdentity+ command describe properties of the desired
  AIK pair.

\item The TPM responds to the \verb+TPM_MakeIdentity+ command by
  generating a new AIK wrapped by SRK, installing the new AIK, and
  returning $\public{AIK}$, signed by the TPM using $\private{AIK}$ --
  $\sign{\public{AIK}}{AIK}$.

  WHAT IS CR???  IS CR THE SIGNED KEYS?

  \emph{The new AIK is a key wrapped by SRK.  The only way that
    $\sign{\public{AIK}}{AIK}$ can be created is in the presence of
    the TPM that generated AIK.}

\item The user software sends the certified public EK
  ($\sign{EK}{AIK}$) and the signed public \public{AIK}
  ($\sign{AIK}{AIK}$) obtained from the TPM to the Privacy CA and
  requests that it certify $\public{AIK}$.

\item The Privacy CA uses $\public{AIK}$ to check the signature on EK
  and then determines if it has been revoked.  It then uses
  $\public{AIK}$ to check the signature on itself.  If AIK signature
  check and EK signature check succeed, the Privacy CA knows the AIK
  and EK came from the same TPM.  If both checks are successful, the
  Privacy CA signs the public AIK and encrypts the certificate with a
  fresh session key.  It they encrypts both the new session key and
  the AIK with EK and returns both to the user software.

\item The user software uses the \verb+TPM_ActivateIdentity+ command
  to decyrpt the session key ($\encrypt{K}{EK}$).  Only the TPM
  associated with EK can decrypt the session key.  The user software
  then turn decrypts the signed AIK ($\encrypt{\sign{AIK}{CA}}{K}$)
  using $K$.  The $\public(AIK)$ signed by the Privacy CA
  ($\sign{\public{AIK}}{CA}$) can only be obtained in the presence of
  the TPM associated with EK.  \emph{This is exactly the result that
    we want}.

\item The user software requests a quote from the TPM using the
  \verb+TPM_Quote+ command and the nonce, $n$, sent by the appraiser.
  The TPM produces the quote, signed using $\private{AIK}$.
  $\sign{\public{AIK}}{CA}$ is also signed with $\private{AIK}$.  The
  user software sends the signed, certified AIK and signed quote back
  to the appraiser.

\item The appraiser analyzes the signed blobs received from the user
  software as follows: 
  
  \begin{enumerate}
  \item $\public{AIK}$ is used to check the signature of
    $\sign{\sign{\public{AIK}{CA}}}{AIK}$ --- The certified $AIK$ was
    signed by the same TPM as the quote.
    
  \item $\public{CA}$ is used to check the signature of
    $\sign{\public{AIK}}{CA}$ --- The Privacy CA has certified the AIK
    as coming from a legitimate TPM.
  
  \item $\public{AIK}$ is used to check the quote signature --- The
    quote sent was signed by the TPM associated with the AIK.
    
  \item $n$ is checked against the nonce sent with the original
    request --- The TPM quote is fresh.
    
  \item PCRs from the quote are compared with expected values --- The
    remote system is configured as expected.
  \end{enumerate}
\end{enumerate}

\begin{figure}
  \centering
  \begin{sequencediagram}
    \newthread[white]{tpm}{TPM}
    \newinst[3]{user}{User Software}
    \newinst[3]{ca}{Certificate Authority}
    \newinst[1]{a}{Appraiser}

    \begin{call}{a}{Attestation Request with $n$ and $PCR$ mask}{user}{$(\sign{\sign{\public{AIK}}{CA}}{AIK},\sign{n,PCRDigest}{AIK})$}
      \begin{call}{user}{\textsf{MakeIdentity}}{tpm}{$\sign{\public{AIK}}{AIK}$,CR}
        \begin{callself}{tpm}{Generate new AIK}{$\wrap{AIK}{SRK}$}
        \end{callself}
        \begin{callself}{tpm}{Install $\wrap{AIK}{SRK}$}{}
        \end{callself}
      \end{call}
      \begin{call}{user}{$\public{\sign{\public{EK}}{AIK}},\sign{\public{AIK}}{AIK}$}{ca}{$\encrypt{\sign{\public{AIK}}{CA}}{K},\encrypt{K,\public{AIK}}{EK}$}
        \begin{callself}{ca}{Check \public{AIK}}{}
        \end{callself}     
        \begin{callself}{ca}{Check \public{EK}}{}
        \end{callself}
      \end{call}
      \begin{call}{user}{$\mathsf{ActivateIdentity}(\encrypt{K,AIK}{EK})$}{tpm}{$K$}
      \end{call}
      \begin{callself}{user}{Decrypt $\encrypt{\sign{\public{AIK}}{CA}}{K}$}{$\sign{\public{AIK}}{CA}$}
      \end{callself}     
      \begin{call}{user}{$\mathsf{Quote}(n,PCRMask)$}{tpm}{$\sign{n,PCRDigest}{AIK}$}
      \end{call}
    \end{call}
  \end{sequencediagram}
  \caption{Sequence Diagram for the Privacy CA protocol as described
    by Ryan}
  \label{fig:ryan-ca}
\end{figure}

\subsection{TCG Documentation Approach}

The TCG protocol, shown in figure~\ref{fig:tcg-ca}, is documented in
the TPM technical documentation by way of describing the TPM command
set provided~\citep{---::TCG-TPM-Specifi}.  At this time, the
distinction is the \verb+TPM_MakeIdentity+ command returning a signed
public $\public{EK}$ in addition to the signed $\public{AIK}$.  If we
can determine that this is the same as the certification request
($CR$), then the two protocols are basically the same.

\begin{figure}
  \centering
  \begin{sequencediagram}
    \newthread[white]{tpm}{TPM}
    \newinst[3]{user}{User Software}
    \newinst[3]{ca}{Certificate Authority}
    \newinst[1]{a}{Appraiser}

    \begin{call}{a}{Attestation Request with $n$ and $PCR$ mask}{user}{$(\sign{\sign{\public{AIK}}{CA}}{AIK},\sign{n,PCRDigest}{AIK})$}
      \begin{call}{user}{\textsf{MakeIdentity}}{tpm}{$\sign{\public{AIK}}{AIK}$,$\sign{EK}{AIK}$}
        \begin{callself}{tpm}{Generate new AIK}{$\wrap{AIK}{SRK}$}
        \end{callself}
        \begin{callself}{tpm}{Install $\wrap{AIK}{SRK}$}{}
        \end{callself}
      \end{call}
      \begin{call}{user}{$\public{\sign{\public{EK}}{AIK}},\sign{\public{AIK}}{AIK}$}{ca}{$\encrypt{\sign{\public{AIK}}{CA}}{K},\encrypt{K,\public{AIK}}{EK}$}
        \begin{callself}{ca}{Check \public{AIK}}{}
        \end{callself}     
        \begin{callself}{ca}{Check \public{EK}}{}
        \end{callself}
      \end{call}
      \begin{call}{user}{$\mathsf{ActivateIdentity}(\encrypt{K,AIK}{EK})$}{tpm}{$K$}
      \end{call}
      \begin{callself}{user}{Decrypt $\encrypt{\sign{\public{AIK}}{CA}}{K}$}{$\sign{\public{AIK}}{CA}$}
      \end{callself}     
      \begin{call}{user}{$\mathsf{Quote}(n,PCRMask)$}{tpm}{$\sign{n,PCRDigest}{AIK}$}
      \end{call}
    \end{call}
  \end{sequencediagram}
  \caption{Sequence Diagram for the Privacy CA protocol as described
    by Ryan}
  \label{fig:tcg-ca}
\end{figure}

\section{Direct Anonymous Attestation}

\nocite{---::TCG-TPM-Specifi,Ryan:09:Introduction-to}

\section{Glossary}

\begin{description}
\item[$\sign{M}{K}$] --- M signed with private K.
\item[$\encrypt{M}{K}$]  --- M encrypted with public K.
\end{description}

\bibliography{attestation}

\end{document}
