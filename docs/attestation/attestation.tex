\documentclass[10pt]{article}

%\usepackage{hyperref}
\usepackage{fullpage}
\usepackage{alltt}
\usepackage{natbib}
\usepackage{graphicx}
\usepackage{url}
\usepackage{fancyhdr}
\usepackage{tikz}
\usetikzlibrary{arrows,shadows}
\usepackage[underline=false]{pgf-umlsd}
\usepackage{trust}
\pagestyle{fancy}

\lhead{}
\rhead{}
\lfoot{\copyright The University of Kansas, 2012}
\cfoot{\thepage}


\newtheorem{conjecture}{Conjecture}
\newtheorem{obligation}{Obligation}
\newtheorem{definition}{Definition}


\usepackage{ifthen}
\newboolean{submission}  %%set to true for the submission version
\setboolean{submission}{false}
%\setboolean{submission}{true}
\ifthenelse
{\boolean{submission}}
{ \newcommand{\todo}[1]{ } } % hide todo
{ \newcommand{\todo}[1]{ % show todo
   \marginpar{\raggedright\footnotesize{#1}}
               }}

\parskip=\medskipamount
\parindent=0pt

\bibliographystyle{abbrvnat}

\title{Attestation Protocols: A Tutorial Introduction}
\author{Perry Alexander and Brigid Halling \\ Information and
  Telecommunication Technology Center \\
  The University of Kansas \\
  \url{{palexand,bhalling}@ku.edu}}

\begin{document}

\maketitle
\tableofcontents
\listoffigures
\listoftables

\begin{abstract}
  This document is intended to provide a tutorial overview of the
  basic attestation protocols used by a TPM.
\end{abstract}

\section{Introduction}

\section{Privacy Certificate Authority Based Attestation}

There are two versions of the Privacy CA attestation process, one
documented by~\citet{Ryan:09:Introduction-to} and the other documented
in the TCG specification \citep{---::TCG-TPM-Specifi}.  They are not
dramatically different, but should be reconciled.

\subsection{The Ryan Approach}

The Ryan protocol, shown in figure~\ref{fig:ryan-ca}, is documented in
his invaluable technical report~\cite{Ryan:09:Introduction-to}.  We
have enhanced it here to include more explicit interaction between the
appraiser and the user software.  

\begin{enumerate}
  \parskip=0pt\itemsep=0pt
\item An appraiser sends an attestation request indicating what PCRs
  are needed using a $PCR$ mask along with a nonce, $n$.\footnote{This
  request is not formally documented, but its specifics are not
  critical for this discussion.}

\item The user software receives the request and requests a fresh AIK
  from the TPM, wrapped with the TPM's SRK using the
  \verb+TPM_MakeIdentity+ command.  Parameters to the
  \verb+TPM_MakeIdentity+ command describe properties of the desired
  AIK pair and include a digest of the CA and Label identifying the
  target CA ($CA_d$).  

\item The TPM responds to the \verb+TPM_MakeIdentity+ command by
  generating a new AIK wrapped by SRK and returns the new wrapped AIK
  along with $idContents$ containing $CA_d$ and $\public{AIK}$ signed
  by $\private{AIK}$.

  \emph{The new AIK is a key wrapped by SRK and can only be installed
    on the TPM associated with SRK.  Thus, the only way that
    $\sign{CA_d,\public{AIK}}{AIK}$ can be created is in the presence
    of the TPM that generated AIK.}

\item The user software sends a certified public EK ($\sign{EK}{AIK}$)
  and the $idContents$ ($\sign{CA_d,\public{AIK}}{AIK}$) obtained from
  the TPM to the Privacy CA and requests that it certify
  $\public{AIK}$.

  \emph{It is not clear where the certified public EK is obtained, but
    the public EK is available to the CA in many different fashions.}

\item The Privacy CA uses $\public{AIK}$ to check the signature on
  $\public{EK}$ and then determines if the $EK$ has been revoked.  It
  then uses $\public{AIK}$ to check the signature on
  $idContents$. \emph{If $idContents$ signature check and EK signature
    check succeed, the Privacy CA knows the $\public{AIK}$, $CA_d$ and
    $\public{EK}$ came from the same TPM.}  It then uses its own
  public key and name to regenerate $CA_d$ and compare with the signed
  $CA_d$ sent to it.  \emph{If the generated digest is the same as the
    received digest, the CA knows the certification request is
    intended for it.} If all three checks are successful -- EK
  signature, $idContents$ signature, and $CA_d$ generation -- the
  Privacy CA signs the public AIK and encrypts the AIK certificate
  with a fresh session key.  It they encrypts both the session key and
  the certified $\public{AIK}$ with $\public{EK}$ and returns both to
  the user software.  \emph{Only the TPM with $\private{EK}$ can
    decrypt this package, ensuring that only the TPM that generated
    the request can decrypt the response.}

\item The user software uses the \verb+TPM_ActivateIdentity+ command
  to decyrpt the session key ($\encrypt{K}{EK}$).  \emph{Only the TPM
    associated with $\private{EK}$ can decrypt the session key.}  The
  user software then in turn decrypts the signed AIK
  ($\encrypt{\sign{AIK}{CA}}{K}$) using $K$.  \emph{The $\public(AIK)$
    signed by the Privacy CA ($\sign{\public{AIK}}{CA}$) can only be
    obtained in the presence of the TPM associated with EK.}

\item The user software requests a quote signed with $\private{AIK}$
  from the TPM using the \verb+TPM_Quote+ command and the nonce, $n$,
  sent by the appraiser.  The TPM produces the quote, signed using
  $\private{AIK}$.  \emph{This ensures the quote comes from the holder
    of $\private{AIK}$.}  $\sign{\public{AIK}}{CA}$ is also signed
  with $\private{AIK}$.  \emph{This ensures the certified public AIK
    comes from the only TPM that could decrypt it.}  The user software
  sends the signed, certified AIK and signed quote back to the
  appraiser.\footnote{The details of this communication are not
    specified.}

\item The appraiser analyzes the signed blobs received from the user
  software as follows: 
  
  \begin{enumerate}
  \item $\public{AIK}$ is used to check the signature of
    $\sign{\sign{\public{AIK}}{CA}}{AIK}$ --- The $AIK$ was signed by
    the TPM where the $AIK$ is installed.
    
  \item $\public{CA}$ is used to check the signature of
    $\sign{\public{AIK}}{CA}$ --- The Privacy CA has certified the AIK
    is installed in a legitimate TPM.  The session key encrypting the
    certificate can only be decrypted by the TPM making the
    certification request.
  
  \item $\public{AIK}$ is used to check the quote signature --- The
    quote sent was signed by the TPM where the AIK is installed and
    the same TPM that generated the signature on
    $\sign{\sign{\public{AIK}}{CA}}{AIK}$
      
  \item $n$ is checked against the nonce sent with the original
    request --- The TPM quote is fresh, avoiding a replay attack using
    a cached quote.
    
  \item PCRs from the quote are compared with expected values --- The
    remote system is configured as expected.
  \end{enumerate}
\end{enumerate}

\begin{figure}
  \centering
  \begin{itemize}
    \parskip=0pt\itemsep=0pt
  \item $CA_d \equiv$ CA public key and label digest
  \item $SRK_h \equiv$ SRK key handle
  \item $AIK_h \equiv$ AIK key handle
  \item $AIK_d \equiv$ public AIK key digest    
  \item $PCR_m \equiv$ mask specifying target PCRs for a quote    
  \item $PCR_d \equiv$ digest of PCR values
  \end{itemize}

  \begin{footnotesize}
  \begin{sequencediagram}
    \newthread[white]{tpm}{TPM}
    \newinst[5]{user}{User Software}
    \newinst[3]{ca}{Certificate Authority}
    \newinst[0.25]{a}{Appraiser}

    \begin{call}{a}{Attestation Request with $n$ and $PCR$ mask}{user}{$(\sign{\sign{\public{AIK}}{CA}}{AIK},\sign{n,PCR_d}{AIK})$}
      \begin{call}{user}{$\mathtt{MakeIdentity}(CA_d)$}{tpm}{$\wrap{AIK}{SRK},\sign{CA_d,AIK}{AIK}$}
      \end{call}
      \begin{call}{user}{$\mathtt{LoadKey2}(\wrap{AIK}{SRK},SRK_h)$}{tpm}{$AIK_h$}
      \end{call}
%{$\sign{\public{AIK}}{AIK}$,\sign{\public{EK}}{AIK}}
      \begin{call}{user}{$(\sign{\public{EK}}{AIK},\sign{\public{CA_d,AIK}}{AIK})$}{ca}{$(\encrypt{\sign{\public{AIK}}{CA}}{K},\encrypt{K,\public{AIK}}{EK})$}
        \begin{callself}{ca}{Check \public{AIK}}{}
        \end{callself}     
        \begin{callself}{ca}{Check \public{EK}}{}
        \end{callself}
      \end{call}
      \begin{call}{user}{$\mathtt{ActivateIdentity}(AIK_h,\encrypt{K,AIK_d}{EK})$}{tpm}{$K$}
      \end{call}
      \begin{callself}{user}{Decrypt $\encrypt{\sign{\public{AIK}}{CA}}{K}$}{$\sign{\public{AIK}}{CA}$}
      \end{callself}     
      \begin{call}{user}{$\mathtt{Quote}(n,AIK_h,PCR_m)$}{tpm}{$\sign{n,PCR_d}{AIK}$}
      \end{call}
    \end{call}
  \end{sequencediagram}
  \end{footnotesize}
  \caption{Sequence Diagram for the Privacy CA protocol as described
    by Ryan}
  \label{fig:ryan-ca}
\end{figure}

\subsection{TCG Documentation Approach}

The TCG protocol is documented in the TPM technical documentation by
way of describing the TPM command set
provided~\citep{---::TCG-TPM-Specifi}.  At this time, the distinction
is the \verb+TPM_MakeIdentity+ command returning a signed public
$\public{EK}$ in addition to the signed $\public{AIK}$.  If we can
determine that this is the same as the certification request ($CR$),
then the two protocols are basically the same.

\newpage

\section{Direct Anonymous Attestation}

\nocite{---::TCG-TPM-Specifi,Ryan:09:Introduction-to}

\begin{figure}
  \centering
  \begin{footnotesize}
  \begin{sequencediagram}
    \newthread[white]{tpm}{TPM}
    \newinst[2]{user}{User Software}
    \newinst[2]{issuer}{Issuer}
    \newinst[2]{verifier}{verifier}
    \begin{call}{verifier}{Attestation Request with $n$ and $PCR$
        mask}{user}{$(\sign{\public{AIK}}{AIK},\sign{n,PCR_d}{AIK}),\sign{\public{AIK},\public{V},n}{DAA}$}
      \begin{call}{user}{\emph{request} $\public{DAA}$}{tpm}{$\public{EK},\public{DAA}$}
      \end{call}
    \begin{call}{user}{$\public{EK}$,$\public{DAA}$}{issuer}{$\sign{\public{DAA}}{IS}$}
    \end{call}
    \begin{call}{user}{$\mathtt{Quote}(n,AIK_h,PCR_m)$}{tpm}{$\sign{n,PCR_d}{AIK}$}
    \end{call}
  \end{call}
  \end{sequencediagram}
  \end{footnotesize}
  \caption{Basic DAA protocol execution **DRAFT**.}
  \label{fig:daa}
\end{figure}

Don't actually deliver the DAA certificate signed by the issuer.
Instead, a proof that:

\begin{itemize}
\item User generated signature by $DAA$ on $AIK$, $V$, and $n$
  $(\sign{\public{AIK},\public{V},n}{DAA})$.  $V$ identifiers the
  target verifier.  $AIK$ is the identity being verified as coming
  from a legitimate TPM and used for the quote.  $n$ is a nonce to
  ensure freshness.
\item User possess the certified DAA from the issuer
  $(\sign{DAA}{IS})$, but does not send it to the verifier.
\end{itemize}

\subsection{Camenisch-Lysyanskaya Signature Scheme}

Define the public key of the DAA signature issuer as:

\[\public{IS} \equiv (n,a,b,d)\]

where $n$ is an RSA modulus.  Then define the DAA signature on message
$x$:

\[\sign{x}{DAA} \equiv (c,e,s)\]

such that:

\begin{equation}
  c^e = a^xb^sd \bmod n
  \label{eq:sig-check}
\end{equation}

The DAA signature gives $c$, $e$, and $s$ while the DAA public key
gives $n$, $a$, $b$, and $d$.  Knowing $x$, its DAA signature, and the
DAA public key for the signer allows checking
equation~\ref{eq:sig-check}.  However, we don't want to hand out the
signature on $x$.  Can we instead prove that we have it without
handing it out?

For DAA, the public key associated with a TPM is

\[DAA \equiv a^x \bmod n\]

where $x$ is the secret key of the TPM.  Is $x$ equal to
$\private{AIK}$ or $\private{EK}$?

The protocol convinces the verifier of possession of certificate on
secret message.  Specifically, Prover wants to convince Verifier that
they know $x$ such that $y=a^x$ where the verifier learns only $y$ and
$a$:

\begin{figure}[h]
  \begin{itemize}
    \itemsep=0pt\parskip=0pt
  \item $r$ is random
  \item $c$ is random
  \item Verifier checks $t=y^ca^s$ (I think)
  \end{itemize}
  \begin{sequencediagram}
    \newthread[white]{prover}{Prover}
    \newinst[3]{verifier}{Verifier}
    \begin{call}{prover}{$t=a^r$}{verifier}{$c$}
    \end{call}
    \mess{prover}{$s=r-cx$}{verifier}
  \end{sequencediagram}
  \caption{Proving knowledge of certificate on secret message}
  \label{fig:proof-of-knowledge}
\end{figure}

Prover wants to convince verifier that they know $x_1$ and $x_2$
such that $y=a^{x_1}b^{x_2}$ where the verifier learns only $y$, $a$
and $b$:

\begin{figure}[h]
  \begin{itemize}
    \itemsep=0pt\parskip=0pt
  \item $r_1$, $r_2$ are random
  \item $c$ is random
  \item Verifier checks $t=y^ca^{s_1}b^{s_2}$ (I think)
  \end{itemize}
  \begin{sequencediagram}
    \newthread[white]{prover}{Prover}
    \newinst[4]{verifier}{Verifier}
    \begin{call}{prover}{$t=a^{r_1}b^{r_2}$}{verifier}{$c$}
    \end{call}
    \mess{prover}{$s_1=r_1-cx_1$,$s_2=r_2-cx_2$}{verifier}
  \end{sequencediagram}
  \caption{Proving knowledge of certificate on secret message}
  \label{fig:proof-of-knowledge}
\end{figure}

Recall $x$ and $(c,e,s)$ such that $c^e=a^xb^sd \bmod n$.

\begin{itemize}
  \item blind cert: compute $c'=cb^{s'} \bmod n$ with random $s'$.
    $c'^e=a^xb^{s^*}d \bmod n$
  \item send $c'$ to verifier
  \item prove knowledge of $x$, $e$, $s^*$ such that
    $d=c'^ea^{-x}b^{-s^*} \bmod n$
\end{itemize}

\section{Glossary}

\begin{description}
\item[$\sign{M}{K}$] --- M signed with private K.
\item[$\encrypt{M}{K}$]  --- M encrypted with public K.
\end{description}

\nocite{Camenisch:03:A-Signature-Sch}

\bibliography{attestation}

\end{document}
